\documentclass[6pt]{article}
\usepackage{ctex} 
\usepackage{amsmath}
\usepackage{graphicx}
\usepackage{geometry}
\usepackage{listings}
\usepackage{parskip}
\usepackage{xcolor}
\usepackage{booktabs}
\usepackage{array}
\usepackage{amsmath} 
\usepackage{amssymb} 

\geometry{
a4paper,
total={170mm,257mm},
left=20mm,
top=10mm,
right=5mm,
}

\usepackage{graphicx}
\begin{document}

\Large
\begin{center}
\noindent{\bf DATA 468: Applied Stochastic Process }\\
\vspace{40pt}
\bf Instructor: Dr.Zakir Ullah (zakir@arizona.edu).\\
\vspace{40pt}
\noindent{\bf Department of Mathematics, University of Arizona, USA}
\end {center}

\vspace{40pt}
\begin{center}

\bf Final Exam Spring Semester, Date: 30$^{th}$ May 2025, Total marks: 100, Duration: (14:00-16:00).
\end{center}
\vspace{5pt}
%\medskip\hrule
\vspace{20pt}

\bf Instructions \\
1. Please leave all personal belongings at the front of the classroom. Do not begin the exam until instructed to do so.\\
2. Talking or looking at other students' exams is strictly prohibited. If you require assistance, please raise your hand and speak with the instructor.\\
3. The use of digital devices, including phones and computers, is not allowed during the exam.\\
4. Ensure that you attempt all questions, clearly encircle or tick your selected answer.


\vspace{10pt}

\noindent{\bf Name:----------------------------------Class-----------------------------------}\\
\vspace{10pt}

\noindent{\bf Student ID:--------------------------------------------------------------------- }\\
\vspace{10pt}

\noindent{\bf Date:---------------------------------------------------------------------------- }\\

\vspace{100pt}
\large

1. Choose the right answer. (10 points)

\vspace{10pt}

i). A state is called \hspace{30pt} state if, once the chains enters the state, it remains there forever

(a) null recurrent (b) limiting (c) stationary (d) absorbing

ii). A \hspace{30pt} distribution doesn’t change over time under the Markov process.

(a) transient (b) stationary (c) initial (d) None of them

iii). A stationary distribution is a probability distribution over the states of a Markov chain such that, if the chain \hspace{30pt} in it, it will remain in it all future time steps.

(a) stuck (b) starts (c) end (d) none of these 

iv). The transient distribution describes the probability distribution of the states at a particular\hspace{30pt} time step.

(a) Infinite (b) finite (c) one (d) none of these 

v). Limiting distribution exists if a Markov chain is irreducible, positive recurrent, and\hspace{30pt}

(a) periodic (b) has absorbing states (c) aperiodic (d) None of these

\vspace{10pt}

2. Mark each statement as True or False. (10 points)

\vspace{10pt}

i). A limiting distribution of the Markov chain converges as time goes to infinity, regardless of the starting distribution.

(a) True (b) False

ii). Each column of any transition matrix ($P$) should sum to $1$.

(a) True (b) False 

iii). A finite set of possible states, often represented as {1, 2, ...}. 

(a) True (b) False 

iv). A positive recurrent state has an infinite expected time to return.

(a) True (b) False 

v). For a discrete Markov Chain, the initial distribution is a probability distribution over states at time $n$=0.

(a) True (b) False 

\vspace{10pt}

3. Draw a Transition diagram for the following matrix and identify the transient, recurrent, and absorbing states in it. (10 points)

\[
P = \begin{bmatrix}
0.4 & 0.3 & 0.3 & 0.0 & 0.0 \\
0.0 & 0.5 & 0.0 & 0.5 & 0.0 \\
0.5 & 0.0 & 0.5 & 0.0 & 0.0 \\
0.0 & 0.5 & 0.0 & 0.5 & 0.0 \\
0.0 & 0.3 & 0.0 & 0.3 & 0.4
\end{bmatrix}
\]

\vspace{10pt}

\vspace{800pt}

4. Consider a website with 3 different types of pages: $\{A: Home, B:Articles, C:PDF\}$, the Home page has a link to the Articles page and Articles page has a link back to Home page. Then the Article page has hyperlinks to published articles, clicking any of the hyperlinks opens a PDF document. When the user clicks the hyperlinks it opens the PDF version of the published article, which does not have any links to the previous pages. (30 points)

\[
P=
\left[ {\begin{array}{ccc}
0 & 1 & 0\\
0.5 & 0 & 0.5\\
0 & 0 & 1\\
\end{array} } \right]
\]

When the user initially opens the website, page $A$ is displayed on the screen

\vspace{10pt}

i). Find the periods of all the states and check if the chain is periodic or aperiodic. (6)

\vspace{1500pt}

ii). Find the expected return time to page $A$. (6)

\vspace{1500pt}

iii). Find out the absorbing states of the Markov chain and calculate the absorbing probabilities.(6)

\vspace{1500pt}

iv). Find out if the chain is positive or null-recurrent. (6)

\vspace{1500pt}

v). Findout the stationary distribution if exist. (6)

\vspace{1000pt}

5. Suppose voter preferences for Democrat(D) party, Republican(R) party, and Independents(I) candidates shift around randomly via the transition matrix $P$. If in a previous election, party D got 48\% votes, party R got 42\% votes and party I got 10\% votes out of the total votes cast. (40 points)

\begin{figure}
\centering
\includegraphics[width=14cm]{chain.png}
\caption{Transition Probabilities.}
\label{fig:chain}
\end{figure}


i). Predict the percentage of all the parties' votes distributed during the upcoming election. (6)

\vspace{1000pt}

ii). Check if the limiting distribution for the above Markov Chain exists? (10)

\vspace{1000pt}

iii). If the answer to (ii) is yes, then find the limiting distribution of the above Markov Chain. (8)

\vspace{1000pt}

iv). Does a limiting distribution mean voters stopped switching parties or the overall proportions of voters are stabilized after many election cycles? (6)

\vspace{1000pt}

v). Check if the chain is ergodic? (10)


\vspace{1000pt}


\textbf{Important Formulas}

i). Absorption probabilities

lets $a_i$=P(absorption in a $\mid$ $X_0$=i), where i $\in S$

\[
a_{i} = \sum_{k \in S} a_{k} \cdot P_{ik}
\]
ii) Mean hitting time

\[
h_{ki} = 1+\sum_{k \in S} P_{kj} \cdot h_{ji}, k \ne i
\]

ii) Mean return time 

\[
r_{i} = 1+\sum_{k \in S} P_{ik} \cdot h_{ki}
\]

\end{document}








